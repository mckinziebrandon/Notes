% = = = = = = = = = = = = = = = = = = = = = = = = = = = = = = = = = = = = = = = = = = = = =
% P  R  E  A  M  B  L  E
% = = = = = = = = = = = = = = = = = = = = = = = = = = = = = = = = = = = = = = = = = = = = =
\documentclass[11pt]{article}
\usepackage{amsbsy, amsmath, amssymb, authblk}

%\usepackage{array} 
%\usepackage{algorithm2e}

\usepackage{booktabs, bm}
\usepackage[small,labelfont=bf,up,singlelinecheck=false]{caption}
\usepackage{cancel}
\usepackage{comment}
%\usepackage{fancyhdr}
%\usepackage[default]{lato}
\usepackage[T1]{fontenc}
\usepackage[bottom]{footmisc}
\usepackage{geometry}
\usepackage{graphicx}
\usepackage{hyperref}
%\usepackage[utf8]{inputenc}
%	\inputencoding{latin1}
%	\inputencoding{utf8}
%\usepackage{lettrine}
%\usepackage[sc]{mathpazo}
\usepackage{lmodern} % Nice fonts?
%\usepackage{mathrsfs}
\usepackage{mathtools} 
%\usepackage{marvosym} % silly bullet-point symbols (misc symbols)
%\usepackage{microtype}
\usepackage{minitoc}         % left in case it is needed elsewhere
\setcounter{secttocdepth}{5} % idem
\usepackage{etoc} % for toc before each section.
%\usepackage{multicol}
\usepackage{needspace}
\usepackage{paralist}
%\usepackage{polynom} 			% typesetting polynomial long division
%\usepackage{setspace}
%	\onehalfspacing 
\usepackage{tocloft}
\usepackage{xparse} % DeclareDocumentCommand
\usepackage[compact]{titlesec} 		% compact shrinks whitespace around section headings.
\usepackage{ulem} 				% for strikeout \sout command.
%\usepackage{verbatim}

% Muh packagez :)
\usepackage{../../Packages/MathCommands}
\usepackage{../../Packages/BrandonColors}
\usepackage{../../Packages/BrandonBoxes}
\usepackage{../../Packages/NoteTaker}
\usepackage{../../Packages/CS221}
%\usepackage{../Packages/MachineLearningUtils}


%\usepackage{program}
% DL BOOK CONVENTIONS
\renewcommand\vec[2][]{\bm{#2}_{#1}}

\DeclareDocumentCommand{\slice}
	{ O{t} O{1} m }
	{\vec[\langle #2 \ldots #1 \rangle]{#3}}

\newcommand\myfig[2][0.3\textwidth]{\begin{figure}[h!]\centering\includegraphics[width=#1]{#2}\end{figure}}
\newcommand\myspace[1][]{\vspace{#1\bigskipamount}}
\newcommand\p{\Needspace{10\baselineskip} \noindent}
\newcommand\tlab[1]{\tag{#1}\label{#1}}
\newcommand\Var[1]{\mathrm{Var}\left[#1\right]}


%\usepackage{program}

\usepackage{layout} % Type \layout() anywhere to see values of layout frame.
%\usepackage{showframe} % Displays layout frame on all pages
\usepackage{marginnote}
\renewcommand*{\marginfont}{\scriptsize}

\usepackage{listings}

\definecolor{dkgreen}{rgb}{0,0.6,0}
\definecolor{gray}{rgb}{0.5,0.5,0.5}
\definecolor{mauve}{rgb}{0.58,0,0.82}

\lstset{frame=tb,
	language=Java,
	aboveskip=3mm,
	belowskip=3mm,
	showstringspaces=false,
	columns=flexible,
	basicstyle={\small\ttfamily},
	numbers=none,
	numberstyle=\tiny\color{gray},
	keywordstyle=\color{blue},
	commentstyle=\color{dkgreen},
	stringstyle=\color{mauve},
	breaklines=true,
	breakatwhitespace=true,
	tabsize=3
}

\usepackage{tikz}
\usetikzlibrary{arrows, automata, shapes, snakes, positioning}
\usetikzlibrary{bayesnet}


\titleformat*{\subsubsection}{\small\scshape}
\newcommand\subsub[1]{\Needspace{15\baselineskip}\hrule\subsubsection{#1}\hrule}
\newcommand\matgrad[2]{\nabla_{\mathbf{#2}} #1}

% O{T} means "optional with default value of `T`"
% m means mandatory argument
\DeclareDocumentCommand{\vecseq}
	{ O{n} m }
	{ \{  \vec[1]{#2}, \ldots, \vec[#1]{#2}   \}  }
\DeclareDocumentCommand{\seq}
	{ O{n} m }
	{ \{ #2_1, \ldots #2_#1 \} }
\DeclareDocumentCommand{\dotseq}
	{ O{n} m }
	{ #2_1, \ldots #2_#1 }
	
\newcommand\QA[2]{\item \red{Q}: #1
	\begin{compactitem}
		\item \green{A}: #2
	\end{compactitem}}
	
\newcommand\myref[1]{\purple{[#1]}}

\definecolor{forgeblue}{HTML}{018C9F}
% Gray table borders
\makeatletter
\def\rulecolor#1#{\CT@arc{#1}}
\def\CT@arc#1#2{%
	\ifdim\baselineskip=\z@\noalign\fi
	{\gdef\CT@arc@{\color#1{#2}}}}
\let\CT@arc@\relax
\rulecolor{forgeblue}
\makeatother

%\setlength{\parskip}{1pt}
%\setlength{\columnseprule}{0.1pt}
%\setlength{\columnsep}{0.6cm}
%\setlength\tabcolsep{0.1cm}
\renewcommand{\arraystretch}{1.2}

\makeatletter
\newcommand*\dotp{\mathpalette\dotp@{.5}}
\newcommand*\dotp@[2]{\mathbin{\vcenter{\hbox{\scalebox{#2}{$\m@th#1\bullet$}}}}}
\makeatother

% Title
\title{\vspace{-10mm}\fontsize{24pt}{8pt}\selectfont\textbf{Homework 8: From Language to Logic}\vspace*{-4mm}}
% Author
\author{Brandon McKinzie}
% Date
\date{}

% --------------------------------------------------------------
% --------------------------------------------------------------



\renewcommand\cftsecfont{\small\bfseries}
\renewcommand\cftsubsecfont{\scriptsize}
\renewcommand\cftsubsubsecfont{\scriptsize}

\renewcommand\cftsecafterpnum{\vskip-5pt}
\renewcommand\cftsubsecafterpnum{\vskip-7pt}
\renewcommand\cftsubsubsecafterpnum{\vskip-7pt}

\begin{document}

\maketitle



\section*{Problem 4 Logical Inference}

\textbf{(a)} \question{Some inferences that might look like they're outside the scope of Modus ponens are actually within reach. Suppose the knowledge base contains the following two formulas:
	$$\text{KB} = \{ (A \lor B) \implies C, A  \}$$
	First, convert the knowledge base into conjunctive normal form (CNF). Then apply Modus ponens to derive C. Please show how your knowledge base changes as you apply derivation rules.
}

First we convert to CNF:
\begin{align}
	(A \lor B) \implies  C ~ &\iff ~ \lnot (A \lor B) \lor C \\
	\lnot (A \lor B) ~ &\iff ~ \lnot A \land \lnot B \\ 
	 (\lnot A \land \lnot B) \lor C~ &\iff ~ (
	 	\lnot A \lor C
	 	) \land (
	 		\lnot B \lor C) \\
\end{align}
with the final formula in CNF: $(\lnot A \lor C
) \land (
\lnot B \lor C)$. This means that $\lnot A \lor C$ and $\lnot B \lor C$ are now in our knowledge base. We can apply Modus ponens as follows:
\graybox{
	\lnot A \lor C &\iff A \implies C \\
	\therefore \dfrac{A, A \implies C}{C}
}


\clearpage
\textbf{(b)} \question{Recall that Modus ponens is not complete, meaning that we can't use it to derive everything that's true. Suppose the knowledge base contains the following formulas:
	$$\text{KB} = \{ A \vee B, B \to C, (A \vee C) \to D \}$$
	In this example, Modus ponens cannot be used to derive D, even though D is entailed by the knowledge base. However, recall that the resolution rule is complete.\\
	
	Your task: Convert the knowledge base into CNF and apply the resolution rule repeatedly to derive D.
}

We first convert each formula in the KB to CNF:
\begin{compactitem}
	\item $A \lor B$ is already in CNF.
	\item $B \implies C$ can be written as $\lnot B \lor C$ which is in CNF. 
	\item $(A \lor C) \implies D$ can be written as $\lnot (A \lor C) \lor D \iff (\lnot A \land \lnot C) \lor D \iff (\lnot A \lor D) \land (\lnot C \lor D)$, with the last in CNF.
\end{compactitem}

We now repeatedly apply the resolution rule as follows:
\graybox{
	&\dfrac{A \lor B, \lnot A \lor D}{B \lor D} \\
	&\dfrac{\lnot B \lor C, B \lor D}{C \lor D} \\
	&\dfrac{\lnot C \lor D, C \lor D}{D}
}






\end{document}