% = = = = = = = = = = = = = = = = = = = = = = = = = = = = = = = = = = = = = = = = = = = = =
% P  R  E  A  M  B  L  E
% = = = = = = = = = = = = = = = = = = = = = = = = = = = = = = = = = = = = = = = = = = = = =
\documentclass[11pt]{article}
\usepackage{amsbsy, amsmath, amssymb, authblk}

%\usepackage{array} 
%\usepackage{algorithm2e}

\usepackage{booktabs, bm}
\usepackage[small,labelfont=bf,up,singlelinecheck=false]{caption}
\usepackage{cancel}
\usepackage{comment}
%\usepackage{fancyhdr}
%\usepackage[default]{lato}
\usepackage[T1]{fontenc}
\usepackage[bottom]{footmisc}
\usepackage{geometry}
\usepackage{graphicx}
\usepackage{hyperref}
%\usepackage[utf8]{inputenc}
%	\inputencoding{latin1}
%	\inputencoding{utf8}
%\usepackage{lettrine}
%\usepackage[sc]{mathpazo}
\usepackage{lmodern} % Nice fonts?
%\usepackage{mathrsfs}
\usepackage{mathtools} 
%\usepackage{marvosym} % silly bullet-point symbols (misc symbols)
%\usepackage{microtype}
\usepackage{minitoc}         % left in case it is needed elsewhere
\setcounter{secttocdepth}{5} % idem
\usepackage{etoc} % for toc before each section.
%\usepackage{multicol}
\usepackage{needspace}
\usepackage{paralist}
%\usepackage{polynom} 			% typesetting polynomial long division
%\usepackage{setspace}
%	\onehalfspacing 
\usepackage{tocloft}
\usepackage{xparse} % DeclareDocumentCommand
\usepackage[compact]{titlesec} 		% compact shrinks whitespace around section headings.
\usepackage{ulem} 				% for strikeout \sout command.
%\usepackage{verbatim}

% Muh packagez :)
\usepackage{../../Packages/MathCommands}
\usepackage{../../Packages/BrandonColors}
\usepackage{../../Packages/BrandonBoxes}
\usepackage{../../Packages/NoteTaker}
\usepackage{../../Packages/CS221}
%\usepackage{../Packages/MachineLearningUtils}


%\usepackage{program}
% DL BOOK CONVENTIONS
\renewcommand\vec[2][]{\bm{#2}_{#1}}

\DeclareDocumentCommand{\slice}
	{ O{t} O{1} m }
	{\vec[\langle #2 \ldots #1 \rangle]{#3}}

\newcommand\myfig[2][0.3\textwidth]{\begin{figure}[h!]\centering\includegraphics[width=#1]{#2}\end{figure}}
\newcommand\myspace[1][]{\vspace{#1\bigskipamount}}
\newcommand\p{\Needspace{10\baselineskip} \noindent}
\newcommand\tlab[1]{\tag{#1}\label{#1}}
\newcommand\Var[1]{\mathrm{Var}\left[#1\right]}


%\usepackage{program}

\usepackage{layout} % Type \layout() anywhere to see values of layout frame.
%\usepackage{showframe} % Displays layout frame on all pages
\usepackage{marginnote}
\renewcommand*{\marginfont}{\scriptsize}

\usepackage{listings}

\definecolor{dkgreen}{rgb}{0,0.6,0}
\definecolor{gray}{rgb}{0.5,0.5,0.5}
\definecolor{mauve}{rgb}{0.58,0,0.82}

\lstset{frame=tb,
	language=Java,
	aboveskip=3mm,
	belowskip=3mm,
	showstringspaces=false,
	columns=flexible,
	basicstyle={\small\ttfamily},
	numbers=none,
	numberstyle=\tiny\color{gray},
	keywordstyle=\color{blue},
	commentstyle=\color{dkgreen},
	stringstyle=\color{mauve},
	breaklines=true,
	breakatwhitespace=true,
	tabsize=3
}

\usepackage{tikz}
\usetikzlibrary{arrows, automata, shapes, snakes, positioning}
\usetikzlibrary{bayesnet}


\titleformat*{\subsubsection}{\small\scshape}
\newcommand\subsub[1]{\Needspace{15\baselineskip}\hrule\subsubsection{#1}\hrule}
\newcommand\matgrad[2]{\nabla_{\mathbf{#2}} #1}

% O{T} means "optional with default value of `T`"
% m means mandatory argument
\DeclareDocumentCommand{\vecseq}
	{ O{n} m }
	{ \{  \vec[1]{#2}, \ldots, \vec[#1]{#2}   \}  }
\DeclareDocumentCommand{\seq}
	{ O{n} m }
	{ \{ #2_1, \ldots #2_#1 \} }
\DeclareDocumentCommand{\dotseq}
	{ O{n} m }
	{ #2_1, \ldots #2_#1 }
	
\newcommand\QA[2]{\item \red{Q}: #1
	\begin{compactitem}
		\item \green{A}: #2
	\end{compactitem}}
	
\newcommand\myref[1]{\purple{[#1]}}

\definecolor{forgeblue}{HTML}{018C9F}
% Gray table borders
\makeatletter
\def\rulecolor#1#{\CT@arc{#1}}
\def\CT@arc#1#2{%
	\ifdim\baselineskip=\z@\noalign\fi
	{\gdef\CT@arc@{\color#1{#2}}}}
\let\CT@arc@\relax
\rulecolor{forgeblue}
\makeatother

%\setlength{\parskip}{1pt}
%\setlength{\columnseprule}{0.1pt}
%\setlength{\columnsep}{0.6cm}
%\setlength\tabcolsep{0.1cm}
\renewcommand{\arraystretch}{1.2}

\makeatletter
\newcommand*\dotp{\mathpalette\dotp@{.5}}
\newcommand*\dotp@[2]{\mathbin{\vcenter{\hbox{\scalebox{#2}{$\m@th#1\bullet$}}}}}
\makeatother

% Title
\title{\vspace{-10mm}\fontsize{24pt}{8pt}\selectfont\textbf{Homework 7: Car Tracking}\vspace*{-4mm}}
% Author
\author{Brandon McKinzie}
% Date
\date{}

% --------------------------------------------------------------
% --------------------------------------------------------------



\renewcommand\cftsecfont{\small\bfseries}
\renewcommand\cftsubsecfont{\scriptsize}
\renewcommand\cftsubsubsecfont{\scriptsize}

\renewcommand\cftsecafterpnum{\vskip-5pt}
\renewcommand\cftsubsecafterpnum{\vskip-7pt}
\renewcommand\cftsubsubsecafterpnum{\vskip-7pt}

\begin{document}

\maketitle

Setup:
\begin{compactitem}
	\item We want to drive our car from start to finish (green box). 
	\item World is 2D grid with your car + K others. At each timestep t, you get noisy estimate of dist to other cars. 
	\item Variables: Assume we are only concerned with one other car.
	\begin{compactitem}
		\item $C_t \in \R^2$: actual location of the other car. Unobserved.
		\item $a_t \in \R^2$: your car's position. Observed and controlled by us.
		\item $D_t \sim \mathcal N (||a_t - C_t||, \sigma^2)$
	\end{compactitem}
	\item Goal: Compute $P(C_t \mid D_1, \ldots, D_t)$. 
\end{compactitem}

\section*{Problem 1: Bayesian Network Basics}

\textbf{(a)} \question{Suppose we have a sensor reading for the second timestep, $D_2 = 0$. Compute the posterior distribution $\mathbb P(C_2 = 1 \mid D_2 = 0)$.}


Below is the Bayesian network, where we've observed $D_2=0$:
\begin{drawing}
	\node[blight] (h1) {$C_1$};
	\node[bquery, right=of h1] (h2) {$C_2$};
	\node[blight, right=of h2] (h3) {$C_3$};
	
	\node[blight, below=of h1] (e1) {$D_1$};
	\node[bdark, right=of e1] (e2) {$0$};
	\node[blight, right=of e2] (e3) {$D_3$};
	
	\diredge {h1} {e1, h2};
	\diredge {h2} {e2, h3};
	\diredge {h3} {e3};
\end{drawing}



\begin{align}
	\Prob{C_2=1 \mid D_2 = 0}
		&\propto \Prob{C_2=1, D_2=0} \\
		&= \sum_{c_1} \Prob{C_2=1, D_2=0, c_1} \\
		&= \sum_{c_1} \Prob{c_1} \Prob{C_2=1 \mid c_1} \Prob{D_2=0 \mid C_2=1} \\
		&= 0.5 \sum_{c_1} \Prob{C_2=1 \mid c_1} \Prob{D_2=0 \mid C_2=1} \\
		&= 0.5 \eta \sum_{c_1}  \Prob{C_2=1 \mid c_1}\\
		&= 0.5 \eta (\epsilon + (1 - \epsilon)) \\
		&=0.5 \eta \\
	\Prob{D_2  = 0}
		&= \sum_{c_2} \Prob{D_2=0, c_2} \\
		&= \sum_{c_2} \Prob{D_2=0 \mid c_2} \sum_{c_1} \Prob{c_2 \mid c_1} \Prob{c_1} \\
		&= (1 - \eta) \cdot \left( % d_2=0, c_2=0
			 		0.5 (1 - \epsilon) % c_1=0
			 		+ 0.5 \epsilon
		\right) + 
		\eta \cdot \left( % d_2=0, c_2=1
			0.5 \epsilon + 0.5 ( 1 - \epsilon)
		\right) \\
		&= 1 \\
	\therefore \Prob{C_2=1 \mid D_2=0} &= \frac{\Prob{C_2=1, D_2=0}}{\Prob{D_2=0}} = 0.5 \eta 
\end{align}


\clearpage

\textbf{(b)} \question{Compute $\mathbb P(C_2=1 \mid D_2=0, D_3=1)$}

Now our Bayesian network looks like:

\begin{drawing}
	\node[blight] (h1) {$C_1$};
	\node[bquery, right=of h1] (h2) {$C_2$};
	\node[blight, right=of h2] (h3) {$C_3$};
	
	\node[blight, below=of h1] (e1) {$D_1$};
	\node[bdark, right=of e1] (e2) {$0$};
	\node[bdark, right=of e2] (e3) {$1$};
	
	\diredge {h1} {e1, h2};
	\diredge {h2} {e2, h3};
	\diredge {h3} {e3};
\end{drawing}


\begin{align}
	\Prob{C_2 = 1 \mid D_2 = 0, D_3 = 1}
		&\propto \Prob{C_2=1, D_2=0, D_3=1} \\
		&= \sum_{c_1} \sum_{c_3} \Prob{D_3=1 \mid c_3} \Prob{c_3 \mid C_2=1} \Prob{C_2=1 \mid c_1} \Prob{c_1} \Prob{D_2=0 \mid C_2=1}
\end{align}











\end{document}